\documentclass[preview,border={5pt,5pt,5pt,5pt}]{standalone}

\usepackage{tikz}
\usepackage[siunitx]{circuitikz}
\usepackage{amsmath}


\begin{document}

\begin{figure}[h!]
  \begin{center}
    \begin{circuitikz}
      \draw (0,0)
      to[american voltage source,v=\SI{12.0}{\volt}] (0,2) % The voltage source
      to[R=\parbox{0em}{\begin{align*} R_1&=\SI{100}{\ohm}\\
                                       I_1&=\SI{16.0}{\milli\ampere}\\
                                       V_1&=\SI{1.60}{\volt}\end{align*}\\\\}] (8,2)
      to[R=\parbox{0em}{\begin{align*} R_2&=\SI{180}{\ohm}\\
                                       I_2&=\SI{16.0}{\milli\ampere}\\
                                       V_2&=\SI{2.88}{\volt}\end{align*}}] (8,0)
      to[R=\parbox{0em}{\begin{align*}\\\\ R_3&=\SI{200}{\ohm}\\
                                       I_3&=\SI{16.0}{\milli\ampere}\\
                                       V_3&=\SI{3.20}{\volt}\end{align*}}] (4,0)
      to[R=\parbox{0em}{\begin{align*}\\\\ R_4&=\SI{270}{\ohm}\\
                                       I_4&=\SI{16.0}{\milli\ampere}\\
                                       V_4&=\SI{4.32}{\volt}\end{align*}}] (0,0);
    \end{circuitikz}
    \caption{Predicted currents and voltages for series circuit.}
  \end{center}
\end{figure}

\end{document}
