\documentclass[preview,border={50pt,5pt,50pt,5pt}]{standalone}

\usepackage{tikz}
\usepackage[siunitx]{circuitikz}
\usepackage{amsmath}


\begin{document}

\setcounter{figure}{2}


\begin{figure}[h!]
  \begin{center}
    \begin{circuitikz}
      \draw (0,0)
      to[american voltage source,v=\SI{12.0}{\volt}] (0,2) % The voltage source
      to[short] (0.9,2)
      to[R=\parbox{0em}{\begin{align*} R_1&=\SI{680}{\ohm}\\
                                       I_1&=\SI{17.6}{\milli\ampere}\\
                                       V_1&=\SI{12.0}{\volt}\end{align*}},*-*] (0.9,0)
      to[short] (0,0);
      \draw (0.9,2)
      to[short] (3.7,2)
      to[R=\parbox{0em}{\begin{align*} R_2&=\SI{680}{\ohm}\\
                                       I_2&=\SI{17.6}{\milli\ampere}\\
                                       V_2&=\SI{12.0}{\volt}\end{align*}},*-*] (3.7,0)
      to[short] (0.9,0);
      \draw (3.7,2)
      to[short] (6.5,2)
      to[R=\parbox{0em}{\begin{align*} R_3&=\SI{620}{\ohm}\\
                                       I_3&=\SI{19.4}{\milli\ampere}\\
                                       V_3&=\SI{12.0}{\volt}\end{align*}},*-*] (6.5,0)
      to[short] (3.6,0);
      \draw (6.5,2)
      to[short] (9.3,2)
      to[R=\parbox{0em}{\begin{align*} R_4&=\SI{750}{\ohm}\\
                                       I_4&=\SI{16.0}{\milli\ampere}\\
                                       V_4&=\SI{12.0}{\volt}\end{align*}}] (9.3,0)
      to[short] (6.5,0);
    \end{circuitikz}
    \caption{Predicted currents and voltages for parallel circuit.}
  \end{center}
\end{figure}

\end{document}
