\documentclass[preview]{standalone}

\usepackage{tikz}
\usepackage[siunitx]{circuitikz}
\usepackage{amsmath}


\begin{document}

\begin{center}
    \Large\textbf{Circuit Lab}\\
    \large\textit{Kade Jorud, Connor LaSota, and Luke Tollefson}
\end{center}

\begin{figure}[h!]
  \begin{center}
    \begin{circuitikz}
      \draw (0,0)
      to[american voltage source,v=\SI{12.0}{\volt}] (0,2) % The voltage source
      to[R=\parbox{0em}{\begin{align*} R_1&=\SI{100}{\ohm}\\
                                       I_1&=\SI{16.0}{\milli\ampere}\\
                                       V_1&=\SI{1.60}{\volt}\end{align*}\\\\}] (8,2)
      to[R=\parbox{0em}{\begin{align*} R_2&=\SI{180}{\ohm}\\
                                       I_2&=\SI{16.0}{\milli\ampere}\\
                                       V_2&=\SI{2.88}{\volt}\end{align*}}] (8,0)
      to[R=\parbox{0em}{\begin{align*}\\\\ R_3&=\SI{200}{\ohm}\\
                                       I_3&=\SI{16.0}{\milli\ampere}\\
                                       V_3&=\SI{3.20}{\volt}\end{align*}}] (4,0)
      to[R=\parbox{0em}{\begin{align*}\\\\ R_4&=\SI{270}{\ohm}\\
                                       I_4&=\SI{16.0}{\milli\ampere}\\
                                       V_4&=\SI{4.32}{\volt}\end{align*}}] (0,0);
    \end{circuitikz}
    \caption{Predicted currents and voltages for series circuit.}
  \end{center}
\end{figure}

\begin{figure}[h!]
  \begin{center}
    \begin{circuitikz}
      \draw (0,0)
      to[american voltage source,v=\SI{12.0}{\volt}] (0,2) % The voltage source
      to[R=\parbox{0em}{\begin{align*} R_1&=\SI{97}{\ohm}\\
                                       I_1&=\SI{16.0}{\milli\ampere}\\
                                       V_1&=\SI{1.57}{\volt}\end{align*}\\\\}] (8,2)
      to[R=\parbox{0em}{\begin{align*} R_2&=\SI{181}{\ohm}\\
                                       I_2&=\SI{16.0}{\milli\ampere}\\
                                       V_2&=\SI{2.92}{\volt}\end{align*}}] (8,0)
      to[R=\parbox{0em}{\begin{align*}\\\\ R_3&=\SI{197}{\ohm}\\
                                       I_3&=\SI{16.0}{\milli\ampere}\\
                                       V_3&=\SI{3.18}{\volt}\end{align*}}] (4,0)
      to[R=\parbox{0em}{\begin{align*}\\\\ R_4&=\SI{266}{\ohm}\\
                                       I_4&=\SI{16.0}{\milli\ampere}\\
                                       V_4&=\SI{4.29}{\volt}\end{align*}}] (0,0);
    \end{circuitikz}
    \caption{Experimental resistances, currents, and voltages for series circuit.}
  \end{center}
\end{figure}


\newpage


\begin{figure}[h!]
  \begin{center}
    \begin{circuitikz}
      \draw (0,0)
      to[american voltage source,v=\SI{12.0}{\volt}] (0,2) % The voltage source
      to[short] (0.9,2)
      to[R=\parbox{0em}{\begin{align*} R_1&=\SI{680}{\ohm}\\
                                       I_1&=\SI{17.6}{\milli\ampere}\\
                                       V_1&=\SI{12.0}{\volt}\end{align*}},*-*] (0.9,0)
      to[short] (0,0);
      \draw (0.9,2)
      to[short] (3.7,2)
      to[R=\parbox{0em}{\begin{align*} R_2&=\SI{680}{\ohm}\\
                                       I_2&=\SI{17.6}{\milli\ampere}\\
                                       V_2&=\SI{12.0}{\volt}\end{align*}},*-*] (3.7,0)
      to[short] (0.9,0);
      \draw (3.7,2)
      to[short] (6.5,2)
      to[R=\parbox{0em}{\begin{align*} R_3&=\SI{620}{\ohm}\\
                                       I_3&=\SI{19.4}{\milli\ampere}\\
                                       V_3&=\SI{12.0}{\volt}\end{align*}},*-*] (6.5,0)
      to[short] (3.6,0);
      \draw (6.5,2)
      to[short] (9.3,2)
      to[R=\parbox{0em}{\begin{align*} R_4&=\SI{750}{\ohm}\\
                                       I_4&=\SI{16.0}{\milli\ampere}\\
                                       V_4&=\SI{12.0}{\volt}\end{align*}}] (9.3,0)
      to[short] (6.5,0);
    \end{circuitikz}
    \caption{Predicted currents and voltages for parallel circuit.}
  \end{center}
\end{figure}

\begin{figure}[h!]
  \begin{center}
    \begin{circuitikz}
      \draw (0,0)
      to[american voltage source,v=\SI{12.0}{\volt}] (0,2) % The voltage source
      to[short] (0.9,2)
      to[R=\parbox{0em}{\begin{align*} R_1&=\SI{680}{\ohm}\\
                                       I_1&=\SI{17.4}{\milli\ampere}\\
                                       V_1&=\SI{11.93}{\volt}\end{align*}},*-*] (0.9,0)
      to[short] (0,0);
      \draw (0.9,2)
      to[short] (3.7,2)
      to[R=\parbox{0em}{\begin{align*} R_2&=\SI{666}{\ohm}\\
                                       I_2&=\SI{17.8}{\milli\ampere}\\
                                       V_2&=\SI{11.93}{\volt}\end{align*}},*-*] (3.7,0)
      to[short] (0.9,0);
      \draw (3.7,2)
      to[short] (6.5,2)
      to[R=\parbox{0em}{\begin{align*} R_3&=\SI{614}{\ohm}\\
                                       I_3&=\SI{19.4}{\milli\ampere}\\
                                       V_3&=\SI{11.93}{\volt}\end{align*}},*-*] (6.5,0)
      to[short] (3.6,0);
      \draw (6.5,2)
      to[short] (9.3,2)
      to[R=\parbox{0em}{\begin{align*} R_4&=\SI{750}{\ohm}\\
                                       I_4&=\SI{15.8}{\milli\ampere}\\
                                       V_4&=\SI{11.93}{\volt}\end{align*}}] (9.3,0)
      to[short] (6.5,0);
    \end{circuitikz}
    \caption{Experimental resistances, currents, and voltages for parallel circuit.}
  \end{center}
\end{figure}


\newpage


\begin{figure}[h!]
  \begin{center}
    \begin{circuitikz}
      \draw (0,8)
      to[american voltage source,v=\SI{12.0}{\volt}] (3,8) % The voltage source
      to[R=\parbox{0em}{\begin{align*} R_2&=\SI{150}{\ohm}\\
                                       I_2&=\SI{30.9}{\milli\ampere}\\
                                       V_2&=\SI{4.64}{\volt}\end{align*}}] (3,6)
      to[R=\parbox{0em}{\begin{align*}\\\\ R_3&=\SI{680}{\ohm}\\
                                       I_3&=\SI{7.7}{\milli\ampere}\\
                                       V_3&=\SI{5.26}{\volt}\end{align*}},*-*] (0,6)
      to[R=\parbox{0em}{\begin{align*} R_1&=\SI{68}{\ohm}\\
                                       I_1&=\SI{30.9}{\milli\ampere}\\
                                       V_1&=\SI{2.10}{\volt}\end{align*}}] (0,8);
      \draw (3,6)
      to[short] (3,4)
      to[R=\parbox{0em}{\begin{align*}\\\\ R_4&=\SI{680}{\ohm}\\
                                       I_4&=\SI{7.7}{\milli\ampere}\\
                                       V_4&=\SI{5.26}{\volt}\end{align*}},*-*] (0,4)
      to[short] (0,6);
      \draw (3,4)
      to[short] (3,2)
      to[R=\parbox{0em}{\begin{align*}\\\\ R_5&=\SI{620}{\ohm}\\
                                       I_5&=\SI{8.5}{\milli\ampere}\\
                                       V_5&=\SI{5.26}{\volt}\end{align*}},*-*] (0,2)
      to[short] (0,4);
      \draw (3,2)
      to[short] (3,0)
      to[R=\parbox{0em}{\begin{align*}\\\\ R_6&=\SI{750}{\ohm}\\
                                       I_6&=\SI{7.0}{\milli\ampere}\\
                                       V_6&=\SI{5.26}{\volt}\end{align*}}] (0,0)
      to[short] (0,2);
    \end{circuitikz}
    \caption{Predicted currents and voltages for complex circuit.}
  \end{center}
\end{figure}

\begin{figure}[h!]
  \begin{center}
    \begin{circuitikz}
      \draw (0,8)
      to[american voltage source,v=\SI{12.0}{\volt}] (3,8) % The voltage source
      to[R=\parbox{0em}{\begin{align*} R_2&=\SI{150}{\ohm}\\
                                       I_2&=\SI{30.8}{\milli\ampere}\\
                                       V_2&=\SI{4.63}{\volt}\end{align*}}] (3,6)
      to[R=\parbox{0em}{\begin{align*}\\\\ R_3&=\SI{682}{\ohm}\\
                                       I_3&=\SI{7.6}{\milli\ampere}\\
                                       V_3&=\SI{5.25}{\volt}\end{align*}},*-*] (0,6)
      to[R=\parbox{0em}{\begin{align*} R_1&=\SI{68}{\ohm}\\
                                       I_1&=\SI{30.8}{\milli\ampere}\\
                                       V_1&=\SI{2.09}{\volt}\end{align*}}] (0,8);
      \draw (3,6)
      to[short] (3,4)
      to[R=\parbox{0em}{\begin{align*}\\\\ R_4&=\SI{668}{\ohm}\\
                                       I_4&=\SI{7.8}{\milli\ampere}\\
                                       V_4&=\SI{5.25}{\volt}\end{align*}},*-*] (0,4)
      to[short] (0,6);
      \draw (3,4)
      to[short] (3,2)
      to[R=\parbox{0em}{\begin{align*}\\\\ R_5&=\SI{615}{\ohm}\\
                                       I_5&=\SI{8.4}{\milli\ampere}\\
                                       V_5&=\SI{5.25}{\volt}\end{align*}},*-*] (0,2)
      to[short] (0,4);
      \draw (3,2)
      to[short] (3,0)
      to[R=\parbox{0em}{\begin{align*}\\\\ R_6&=\SI{750}{\ohm}\\
                                       I_6&=\SI{6.9}{\milli\ampere}\\
                                       V_6&=\SI{5.25}{\volt}\end{align*}}] (0,0)
      to[short] (0,2);
    \end{circuitikz}
    \caption{Experimental resistances, currents, and voltages for complex circuit.}
  \end{center}
\end{figure}

\end{document}
